% This is samplepaper.tex, a sample chapter demonstrating the
% LLNCS macro package for Springer Computer Science proceedings;
% Version 2.21 of 2022/01/12
%
\documentclass[runningheads]{llncs}
\usepackage[inline]{enumitem}
\usepackage[utf8]{inputenc}
\usepackage{amsmath}
\usepackage{geometry}
\geometry{a4paper, left=25mm, right=25mm, top=25mm, bottom=25mm}

% \usepackage{fontspec}
%
\usepackage[T1]{fontenc}
% T1 fonts will be used to generate the final print and online PDFs,
% so please use T1 fonts in your manuscript whenever possible.
% Other font encondings may result in incorrect characters.
%
\usepackage{graphicx}
\usepackage{hyperref}
\usepackage{acronym}
\usepackage{cleveref}
% Workaround for: https://tex.stackexchange.com/q/737204/2388
\makeatletter
\AtBeginDocument
{
    \def\ltx@label#1{\cref@label{#1}}%add braces
    \def\label@in@display@noarg#1{\cref@old@label@in@display{#1}}%remove braces
    \def\label@in@mmeasure@noarg#1{%
    \begingroup%
    \measuring@false%
    \cref@old@label@in@display{#1}%remove braces for multline, see https://tex.stackexchange.com/q/737204/2388
    \endgroup
}%
} %
\makeatother

% Used for displaying a sample figure. If possible, figure files should
% be included in EPS format.
%
% If you use the hyperref package, please uncomment the following two lines
% to display URLs in blue roman font according to Springer's eBook style:
%\usepackage{color}
%\renewcommand\UrlFont{\color{blue}\rmfamily}
%
\newenvironment{inlinelist}{\begin{enumerate*}[label=\emph{(\roman*)}]}{\end{enumerate*}}

\acrodef{AP}{Aggregate Programming}
\acrodef{ac}[AC]{Aggregate Computing}
\acrodef{API}{Application Programming Interface}
\acrodef{dsl}[DSL]{domain-specific language}
\acrodef{fc}[FC]{Field Calculus}
\acrodefplural{FC}[FC]{field calculi}
\acrodef{HOFC}{Higher-Order \acl{FC}}
\acrodef{id}[ID]{identifier}
\acrodef{IoT}{Internet of Things}
\acrodef{JVM}{Java Virtual Machine}
\acrodef{scr}[SCR]{self-organising coordination regions}
\acrodef{TOTA}{Tuples On The Air}
\acrodef{VMC}{Vascular Morphogenesis Controller}
\acrodef{cos}[COS]{Collective Operating System}
\acrodef{cas}[CAS]{Collective Adaptive Systems}

\begin{document}

    \begin{titlepage}
        \centering
        \vspace*{2cm}

        {\scshape\large PhD Programme in Computer Science and Engineering \par}
        \vspace{0.5cm}
        {\scshape\large Cycle XL \par}
        \vspace{0.5cm}

        \rule{\linewidth}{0.4mm} \\ [0.1mm]
        \raisebox{0.2cm}{\rule{\linewidth}{0.8mm}} \\[0.8cm]
        {\large\bfseries PhD Period Abroad Proposal for Marco Polo \par}
        \vspace{0.8cm}
        {\LARGE Integration of Control Barrier Functions for Safety-Critical Guarantees in Aggregate Computing\par}

        \vspace{1.5cm}

        \noindent
        \begin{minipage}[t]{0.3\textwidth}
            \raggedright
            \textbf{Supervisors:}\\[0.5cm]
            Prof. Danilo Pianini\\
            Prof. Mirko Viroli\\
            Prof. Enrico Gallinucci
        \end{minipage}%
        \hfill
        \begin{minipage}[t]{0.3\textwidth}
            \centering
            \textbf{Abroad Supervisor:}\\[0.5cm]
            Prof. Alessandro Vittorio Papadopoulos\\
        \end{minipage}
        \begin{minipage}[t]{0.3\textwidth}
            \raggedleft
            \textbf{PhD Student:}\\[0.5cm]
            Angela Cortecchia
        \end{minipage}

        \vspace{1.5cm}

        \rule{\linewidth}{0.8mm} \\ [0.1pt]
        \raisebox{0.2cm}{\rule{\linewidth}{0.4mm}} \\[1.5cm]

    \end{titlepage}

    \section*{Proposal}

    \subsection*{Context}\label{subsec:context}

    \ac{cas} are ensembles of distributed devices, such as sensors, actuators, and robots,
    that cooperate to achieve global goals under dynamic and unpredictable conditions~\cite{DBLP:conf/huc/Ferscha15}.
%
    Programming such systems is inherently challenging,
    as it must address scalability, device heterogeneity,
    and continuous adaptation to changing contexts.
%
    Several approaches have been proposed to alleviate this complexity.
%
    In particular,
    macroprogramming techniques~\cite{casadei22} raise the level of abstraction,
    allowing developers to specify the behavior of the system at a collective level
    rather than managing interactions of individual devices.
%
    A foundational step in this direction is the \emph{\ac{fc}}~\cite{JLAMP2019,TOCL2019},
    a formal model for programming collective behaviors through \emph{computational fields},
    which map values over space and time and enable composable definitions of self-organizing processes.

    Building on these ideas,
    \ac{ac}~\cite{BealIEEEComputer2015} introduces a functional programming model that supports the compositional development of self-adaptive collective services.
    %
    Here,
    the focus of computation shifts from individual devices to collaborative groups,
    enabling concise and modular expression of complex adaptive behaviors.
    %
    Despite its expressive power, classical approaches often model a single aggregate program.
    %
    Recent work on \emph{aggregate processes}~\cite{Coordination2019-processes}
    extends this view by introducing primitive constructs to manage multiple concurrent aggregate computations.
    %
    This is crucial in practice,
    where multiple processes must coexist, overlap, and evolve in parallel,
    each involving different subsets of devices and spanning distinct spatial and temporal regions.
    %
    Such scenarios call for abstractions and mechanisms that are still only partially addressed in the current state of the art.

    Existing \ac{ac} techniques can adapt to failures, network partitions, and topology changes,
    but they provide only limited support for maintaining non-monotonic global states without centralized control,
    and for organizing the system into subsystems with independent behaviors and goals.
    %
    Morphogenetic algorithms address part of this gap:
    inspired by biological development,
    they create complex structures and patterns from simple local interactions~\cite{DBLP:books/daglib/p/Beal12,DBLP:conf/gecco/MorganC13,DBLP:conf/gecco/ZahadatHS17},
    with applications in modular and swarm robotics or self-assembling systems.

    For global state maintenance,
    gossip algorithms are widely used to exchange information and reach consensus in decentralized settings.
%
    However,
    traditional gossip algorithms are not self-stabilizing:
    after transient faults or topology changes,
    they may fail to converge to a correct state.
%
    In \ac{ac},
    this has been partially addressed with replication mechanisms~\cite{PianiniCoordination2016},
    where multiple gossip instances are run per node.
%
    While this eventually enforces consistency,
    it introduces delays and risks contaminating new computations with obsolete values.

    To address these challenges,
    this project proposes a \textbf{\ac{cos}}:
    an execution and management environment for aggregate processes
    that extends aggregate computing with concepts from operating systems.
%
    Much like a traditional OS manages processes, users, and resources on a single machine~\cite{DBLP:journals/csur/TanenbaumR85},
    the COS is envisioned as middleware capable of coordinating multiple aggregate programs distributed across heterogeneous devices.
%
    The research investigates how OS notions---such as users and permissions, signals and interrupts,
    inter-process communication, and resource management---can be reinterpreted in a collective and distributed context.

    In this vision,
    users and permissions correspond to roles and rights distributed across devices and human actors,
    signals and interrupts affect entire spatial regions instead of single execution threads,
    and inter-process communication goes beyond shared memory or message passing,
    enabling coordination among distributed and possibly non-contiguous regions.
%
    The project also aims to address distributed sensing and actuation,
    where heterogeneous devices can be abstracted into logical collective sensors and actuators,
    thus enabling high-level sensor fusion~\cite{DBLP:journals/arc/Sasiadek02},
    and morphogenetic applications within the \ac{ac} paradigm,
    to support the emergence of complex structures and behaviors from local interactions.

    The proposed \ac{cos} will therefore act as unifying middleware that manages concurrent aggregate processes,
    provides mechanisms for permissions and communication,
    and supports deployment on heterogeneous infrastructures.
%
    By integrating advances in aggregate programming frameworks such as Collektive,
    the COS aims to offer a robust, adaptive,
    and general-purpose environment for programming Collective Adaptive Systems.
%
    The expected contribution is a novel paradigm bridging aggregate computing and distributed operating system principles,
    enabling more powerful, flexible,
    and reliable collective applications in domains such as crowd management, autonomous navigation, and smart cities.

    \section*{Opportunities and Challenges}



    \subsection*{Research Gap}

    \subsection*{Research Plan}


    \bibliographystyle{alpha}
    \bibliography{bibliography}

\end{document}