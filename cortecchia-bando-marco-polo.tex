% This is samplepaper.tex, a sample chapter demonstrating the
% LLNCS macro package for Springer Computer Science proceedings;
% Version 2.21 of 2022/01/12
%
\documentclass[runningheads]{llncs}
\usepackage[inline]{enumitem}
\usepackage[utf8]{inputenc}
\usepackage{amsmath}
\usepackage{geometry}
\geometry{a4paper, left=25mm, right=25mm, top=25mm, bottom=25mm}

% \usepackage{fontspec}
%
\usepackage[T1]{fontenc}
% T1 fonts will be used to generate the final print and online PDFs,
% so please use T1 fonts in your manuscript whenever possible.
% Other font encondings may result in incorrect characters.
%
\usepackage{graphicx}
\usepackage{hyperref}
\usepackage{acronym}
\usepackage{cleveref}
% Workaround for: https://tex.stackexchange.com/q/737204/2388
\makeatletter
\AtBeginDocument
{
    \def\ltx@label#1{\cref@label{#1}}%add braces
    \def\label@in@display@noarg#1{\cref@old@label@in@display{#1}}%remove braces
    \def\label@in@mmeasure@noarg#1{%
    \begingroup%
    \measuring@false%
    \cref@old@label@in@display{#1}%remove braces for multline, see https://tex.stackexchange.com/q/737204/2388
    \endgroup
}%
} %
\makeatother

% Used for displaying a sample figure. If possible, figure files should
% be included in EPS format.
%
% If you use the hyperref package, please uncomment the following two lines
% to display URLs in blue roman font according to Springer's eBook style:
%\usepackage{color}
%\renewcommand\UrlFont{\color{blue}\rmfamily}
%
\newenvironment{inlinelist}{\begin{enumerate*}[label=\emph{(\roman*)}]}{\end{enumerate*}}

\acrodef{AP}{Aggregate Programming}
\acrodef{ac}[AC]{Aggregate Computing}
\acrodef{API}{Application Programming Interface}
\acrodef{dsl}[DSL]{domain-specific language}
\acrodef{fc}[FC]{Field Calculus}
\acrodefplural{FC}[FC]{field calculi}
\acrodef{HOFC}{Higher-Order \acl{FC}}
\acrodef{id}[ID]{identifier}
\acrodef{IoT}{Internet of Things}
\acrodef{JVM}{Java Virtual Machine}
\acrodef{scr}[SCR]{self-organising coordination regions}
\acrodef{TOTA}{Tuples On The Air}
\acrodef{VMC}{Vascular Morphogenesis Controller}
\acrodef{cos}[COS]{Collective Operating System}
\acrodef{cas}[CAS]{Collective Adaptive Systems}
\acrodef{clf}[CLF]{Control Lyapunov Function}
\acrodef{cbf}[CBF]{Control Barrier Function}

\begin{document}

    \begin{titlepage}
        \centering
        \vspace*{2cm}

        {\scshape\large PhD Programme in Computer Science and Engineering \par}
        \vspace{0.5cm}
        {\scshape\large Cycle XL \par}
        \vspace{0.5cm}

        \rule{\linewidth}{0.4mm} \\ [0.1mm]
        \raisebox{0.2cm}{\rule{\linewidth}{0.8mm}} \\[0.8cm]
        {\large\bfseries PhD Period Abroad Proposal for Marco Polo \par}
        \vspace{0.8cm}
        {\LARGE Integration of Control Barrier Functions for Safety-Critical Guarantees in Aggregate Computing\par}

        \vspace{1.5cm}

        \noindent
        \begin{minipage}[t]{0.3\textwidth}
            \raggedright
            \textbf{Supervisors:}\\[0.5cm]
            Prof. Danilo Pianini\\
            Prof. Mirko Viroli\\
            Prof. Enrico Gallinucci
        \end{minipage}%
        \hfill
        \begin{minipage}[t]{0.3\textwidth}
            \centering
            \textbf{Abroad Supervisor:}\\[0.5cm]
            Prof. Alessandro Vittorio Papadopoulos\\
        \end{minipage}
        \begin{minipage}[t]{0.3\textwidth}
            \raggedleft
            \textbf{PhD Student:}\\[0.5cm]
            Angela Cortecchia
        \end{minipage}

        \vspace{1.5cm}

        \rule{\linewidth}{0.8mm} \\ [0.1pt]
        \raisebox{0.2cm}{\rule{\linewidth}{0.4mm}} \\[1.5cm]

    \end{titlepage}

    \section*{Proposal}

    \subsection*{Context}\label{subsec:context}

    \ac{cas} are ensembles of distributed devices—such as sensors,
    actuators, and robots—that cooperate to achieve global goals under dynamic and unpredictable conditions~\cite{DBLP:conf/huc/Ferscha15}.
    %
    Programming such systems is inherently challenging,
    as it requires addressing scalability,
    device heterogeneity,
    and continuous adaptation to changing contexts~\cite{BealIEEEComputer2015}.
    %
    To alleviate this complexity, macroprogramming techniques like \ac{ac} have been proposed to raise the level of abstraction,
    allowing developers to specify the behavior of the system at a collective level rather than managing the interactions of individual devices~\cite{DBLP:journals/csur/Casadei23}.

    In this paradigm,
    the focus of computation shifts from individual nodes to collaborative groups,
    enabling the concise expression of complex self-organizing behaviors~\cite{JLAMP2019}.
    %
    However,
    traditional \ac{ac} approaches primarily ensure eventual consistency through self-stabilization~\cite{TOMACS2018};
    this means the system will recover after a fault, but it does not guarantee safety while the recovery is taking place.

    To address this,
    the research looks toward Control Theory,
    which offers formal methods to specify system behavior.
    %
    Specifically, \ac{clf} provide a certification that a system will continuously move towards a desired goal,
    ensuring stability, while \ac{cbf} act as ``virtual walls'',
    guaranteeing that the system never violates specific safety constraints during its operation~\cite{DBLP:journals/ieeejas/LiWYWH23}.

    The integration of these concepts is envisioned within \ac{cos}:
    a middleware capable of coordinating distributed processes and resources,
    extending the concepts of traditional distributed operating systems~\cite{DBLP:journals/csur/TanenbaumR85}.
    %
    By embedding CLF and CBF principles into the \ac{cos},
    the aim is to create an environment that manages not just the execution of collective code,
    but also the physical safety and stability of the underlying devices.

    \subsection*{Opportunities and Challenges}

    The integration of \acp{clf} and \acp{cbf} into the \ac{ac} framework represents a significant opportunity to enhance the reliability of these systems.

    The primary opportunity lies in formalizing the ``transient behavior'' of collective systems.
    %
    While \ac{ac} provides powerful abstractions for defining \emph{what} the collective state should be (e.g., a spatial pattern),
    Control Theory provides the mathematical tools to certify \textit{how} the system moves toward that state.
    %
    By embedding \acp{clf} (for stability) and \acp{cbf} (for safety) into the execution model,
    it becomes possible to transform the middleware from a system that merely executes instructions into one that guarantees safety.
    %
    This is crucial for a Collective OS that aims to manage physical space and safety margins as critical resources.

    While the integration of \acp{clf} and \acp{cbf} into \ac{ac} opens new opportunities,
    it also presents several challenges that need to be taken into account:
    \begin{itemize}
        \item The ``Local vs. Global'': \ac{ac} operates on collective logic, but safety constraints (like collision avoidance)
            must be enforced locally on device actuators.
            Translating a global safety directive into local barrier functions without centralized control presents a significant theoretical hurdle.
        \item Computational Constraints: The proposed integration requires solving a Quadratic Program (QP) at every control cycle on each device.
            This introduces computational overhead that challenges the lightweight nature typically desired in pervasive computing.
    \end{itemize}

    \subsection*{Research Gap}
    The current state of the art in \acl{ac} excels at \emph{resilience} and \emph{eventual} consistency~\cite{TOMACS2018}.
    %
    Mechanisms such as self-stabilizing algorithms are being developed to recover from transient faults and network topology changes.
    %
    However,
    these approaches generally guarantee that the system will \emph{eventually} converge to a correct state,
    without providing guarantees about the system's behavior \emph{during} the convergence phase.

    In safety-critical contexts, such as multi-robot missions, ``eventual'' correctness is not sufficient.
    %
    If a formation breaks,
    robots might collide or disconnect while reorganizing.
    %
    There is currently a lack of mechanisms within the \ac{ac} paradigm that can strictly enforce safety constraints during the transient phase of self-organization.
    %
    This research aims to fill that gap by integrating the fast convergence of \ac{clf} and the safety guarantees of \ac{cbf} directly into the aggregate programming model.

    \subsection*{Research Plan}
    The research is expected to involve:
    \begin{itemize}
        \item Investigation of integrating \ac{clf} and \ac{cbf} into the \ac{ac} theoretical model,
            specifically by mapping aggregate computational fields to Lyapunov candidates for stability and defining barrier functions for forward invariance;
        \item Specification of safety-critical requirements at the single-device level,
            utilizing local optimization techniques to enforce actuator constraints while pursuing global goals defined at the collective level;
        \item Specification of safety-critical requirements directly at the collective level,
            enabling the declarative expression of global constraints, such as connectivity maintenance or geofencing,
            that are automatically translated into local barriers;
        \item Validation on safety-critical use cases, such as multi-robot systems, to demonstrate that the integrated approach guarantees safety during transient adaptation phases, ensuring the system remains safe distinct from the property of eventual consistency.
    \end{itemize}

    \bibliographystyle{alpha}
    \bibliography{bibliography}

\end{document}